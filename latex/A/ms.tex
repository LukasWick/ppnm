%\pdfoutput=0
\documentclass[twocolumn]{article}
\usepackage{graphicx}
\usepackage{hyperref}
\usepackage{amsmath}
\title{An example of using \LaTeX \ and Gnuplot: Euler spiral}
\author{Lukas Martin Wick}
\begin{document}
\maketitle

An 'Euler spiral' is a curve whose curvature changes linearly with its curve length 
(the curvature of a circular curve is equal to the reciprocal of the radius). 
Euler spirals are also commonly referred to as 'spiros', 'clothoids', or 'Cornu spirals'.

Euler spirals have applications to diffraction computations. They are also widely used as transition curves in railroad engineering/highway engineering
for connecting and transitioning the geometry between a tangent and a circular curve. 
A similar application is also found in photonic integrated circuits. The principle of linear 
variation of the curvature of the transition curve between a tangent and a circular curve defines the geometry of the Euler spiral: 

\begin{itemize}
    \item Its curvature begins with zero at the straight section (the tangent) and increases linearly with its curve length.
    \item Where the Euler spiral meets the circular curve, its curvature becomes equal to that of the latter.
\end{itemize}

Normalized Euler spirals can be expressed as:
\begin{align}
    x &= \int_0^L \cos s^2 ds \\
    y &= \int_0^L \sin s^2 ds
\end{align}
The above is from \url{https://en.wikipedia.org/wiki/Euler_spiral}.
A plot of the normalized Euler spirals can be seen in figure \ref{fig:eulerSpirals}
\begin{figure}[]
    \centering
\input{plot-eulerSpiral.tex}
\caption{Illustration of the Euler spiral function.}
\label{fig:eulerSpirals}
\end{figure}

\end{document}
